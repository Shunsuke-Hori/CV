\documentclass[12pt]{article}

\usepackage[hidelinks]{hyperref}
\usepackage[margin=1in]{geometry} % for margins
\usepackage{makecell} % for \makecell
\usepackage{titlesec}
\titleformat{\section}
  {\normalfont\fontsize{12}{15}\bfseries}{\thesection}{1em}{}[{\titlerule[0.8pt]}]
\usepackage{changepage} % for adjustwidth

\begin{document}

\begin{center}
    \begin{Large}
        \textbf{SHUNSUKE HORI}\\
    \end{Large}
    FACULTY OF ECONOMICS\\
    HITOTSUBASHI UNIVERSITY\\
    Updated on \today
\end{center}

\section*{CONTACT INFORMATION}
Faculty of Economics, Hitotsubashi University\\
2-1 Naka, Kunitachi-shi, Tokyo, Japan, 186-8601\\
Email: \href{mailto:s.hori@r.hit-u.ac.jp}{s.hori@r.hit-u.ac.jp}\\
Website: \url{https://shunsuke-hori.github.io}

\section*{FIELDS OF INTEREST}
Macroeconomics, Macro Labor

\section*{EDUCATION}
\begin{tabular}{@{}lll}
\makecell[tl]{University of California, San Diego\\\hspace{1em}California, U.S.A.}&\makecell[tl]{
Ph.D. in Economics\\
Committee:\\
\hspace{1em}David Lagakos (co-chair)\\
\hspace{1em}Valerie A. Ramey (co-chair)\\
\hspace{1em}Titan Alon\\
\hspace{1em}Alexis Akira Toda\\
\hspace{1em}Munseob Lee\\
\hspace{1em}Ulrike Schaede}&2017--2023\\
\makecell[tl]{Graduate School of University of Tokyo\\\hspace{1em}Tokyo, Japan}& (only finished course work)&2015--2017\\
\makecell[tl]{Keio University\\\hspace{1em}Tokyo, Japan}&\makecell[tl]{B.A. in Economics\\(Summa Cum Laude)}&2011--2015\\
\end{tabular}

\section*{CURRENT POSITIONS}
\begin{tabular}{@{}lll}
Assistant Professor &Faculty of Economics, Hitotsubashi University& 2023--present\\
Project Researcher &University of Tokyo& 2023--present
\end{tabular}

\section*{RELEVANT POSITIONS HELD}
\begin{tabular}{@{}lll}
Research Assistant &University of Tokyo (Prof. Taisuke Nakata)& 2023\\
Research Assistant &UCSD (Prof. Mark Jacobsen)& 2022\\
Intern &International Monetary Fund (Alexis Meyer Cirkel)& 2021\\
Lead Developer &QuantEcon& 2017\\
Research Assistant &University of Tokyo (Prof. Tomohiro Hirano)& 2015--2017
\end{tabular}

\section*{FELLOWSHIP, HONORS, AND AWARDS}
Walter P. Heller Memorial Prize. 2020. University of California, San Diego.\\
Summer 2019 GSA Travel Grant. 2019. University of California, San Diego.\\
Graduate Student research funding. 2019. University of California, San Diego.\\
Japan-IMF Scholarship. 2017-2019. Japan IMF Scholarship Program for Advanced Studies.

\section*{WORKING PAPERS}
``The Secular Decline in Aggregate Hours Worked in Japan: A Reinterpretation'' (Job Market Paper)
\begin{adjustwidth}{1em}{1em}
Abstract: Average hours worked per adult in Japan fell by around one third over the last half-century. The leading explanation focuses on government policies that distort labor supply decisions. This paper provides a new interpretation that stresses the role of income effects in preferences. Through the lens of a model of the market and home sectors and using non-homothetic preferences, I show that the main driver of Japan's decline in hours worked is income effects, rather than labor-market distortions or population aging. The model predicts that average hours of leisure will rise and home-production hours will remain roughly constant over the period, which is consistent with evidence from time-use surveys. An alternative calibration based on only labor market distortions counterfactually predicts that home production hours will rise.
\end{adjustwidth}

\section*{RESEARCH IN PROGRESS}
``Generational war on inflation: Optimal inflation rates for the young and the old'' Joint with Ippei Fujiwara and Yuichiro Waki\\
``Human capital and Technology Diffusion: Expanding the Discussion of Industrial Policy'' Joint with Alexis Meyer Cirkel\\
``Hysteresis in Hours Worked''

\section*{OTHER PROJECTS}
``Quarantine and Its Scar on Labor'' Joint with Asako Chiba and Taisuke Nakata

\section*{TEACHING EXPERIENCE}
\subsubsection*{At Hitotsubashi University}
\begin{tabular}{@{}ll}
Basic Macroeconomics&Fall \& Winter 2023
\end{tabular}

\subsubsection*{As a teaching assistant at UC San Diego}
\begin{tabular}{@{}lll}
Data Analytics for the Social Sciences&Prof. David Arnold&Spring 2023\\
Data Analytics for the Social Sciences&Prof. David Arnold&Spring 2022\\
Short Run Macroeconomics&Prof. James Hamilton&Winter 2022\\
Long Run Macroeconomics&Prof. Titan Alon&Fall 2021\\
Long Run Macroeconomics&Prof. Giacomo Rondina&Spring 2021\\
Long Run Macroeconomics&Prof. Titan Alon&Fall 2020\\
Graduate Macroeconomics C&Prof. Johannes Wieland&Spring 2020\\
Graduate Macroeconomics B&Prof. Giacomo Rondina&Winter 2020
\end{tabular}

\section*{PROFESSIONAL ACTIVITIES}
\subsubsection*{Presentations}
2023 25th Annuak Macro Conference (Osaka)\\
2019 Summer Workshop on Economic Theory (Otaru)
\subsubsection*{Conference Participation}
2019 Economics of Artificial Intelligence\\
2014 Summer Workshop on Economic Theory
\subsubsection*{Referee Service}
\textit{Review of Economic Dynamics}

\section*{COMPUTATIONAL SKILLS}
Language: Julia, Python, Matlab, R, Stata

\section*{OTHER INFORMATION}
Citizenship: Japan\\
Date of birth: May 18th, 1992
\end{document}